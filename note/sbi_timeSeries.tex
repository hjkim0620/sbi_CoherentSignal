\documentclass[11pt,a4paper]{article}

\usepackage{style}

\begin{document}
\title{Simulation-Based Inference: 
\\Time Series of a Coherent Signal}
\author{Hyungjin Kim
\\{\it\small Deutsches Elektronen-Synchrotron DESY, Notkestr. 85, 22607 Hamburg, Germany}}
\maketitle

Consider the following time series
\begin{align}
s(t) = h(t) + n(t)
\end{align}
where $h(t)$ is a coherent signal and $n(t)$ is the noise.
Each of them is characterized by
\begin{align}
\langle n(t_I) n(t_J) \rangle = N_{IJ} = \delta_{IJ} \sigma^2
\end{align}
and 
\begin{align}
h(t) = h_0 \cos(2\pi f t + \varphi)
\end{align}
The index $I$ denotes the discrete time index and $N_{IJ}$ is the noise covariance matrix. The noise is assumed to be white, although this assumption can be easily relaxed. For the generation of simulated time series, we set $\sigma=1$. 

In this note, we generate time series according to the above characterization. We then perform a traditional Bayesian analysis for the estimation of the signal $h(t)$. We also use simulation-based inference technique for the parameter estimation, and compare the performance of two approaches. 

\section{Likelihood}
For the traditional approach, we begin by constructing the likelihood function. The likelihood function is given by
\begin{align}
p( D | \theta ) 
= \frac{1}{\sqrt{ | 2 \pi N | }}
\exp\left[
- \frac{1}{2} (s - h)^T
\cdot N^{-1} \cdot (s - h)
\right]
\end{align}
where we treat the time series as a column vector $s = (s(t_1), \, s(t_2) \, \cdots, s(t_{N_t}) )^T$. The parameter is
\begin{align}
\theta = \{ \sigma, h_0, f , \varphi \}.
\end{align}
We impose flat uniform prior for each of these parameters. Then the posterior can be constructed as
\begin{align}
p(\theta | D)
= p(D | \theta) p(\theta) .
\end{align}


If we model that the underlying noise is white, then the noise covariance matrix is diagonal. The likelihood function can then be simplified as
\begin{align}
P(D | \theta)
= \frac{1}{(2\pi)^{N_t/2} \sigma^{N_t}}  \exp\left[ - \frac{1}{2\sigma^2} \sum_{i=1}^{N_t} (s - h)^2 \right]  .
\end{align}














\bibliographystyle{utphys}
\bibliography{ref}
\end{document}


